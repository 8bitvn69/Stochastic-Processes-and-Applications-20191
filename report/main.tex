\documentclass[12pt,a4paper,oneside]{book}


% \newcommand{\project}{III}
\newcommand{\projectname}{
	Học tăng cường\par
}
% \newcommand{\specialized}{Toán Tin}
% \newcommand{\orientation}{Tin học}
% \newcommand{\studentname}{Phùng Trọng Hiếu}
\newcommand{\teachername}{TS. Nguyễn Thị Ngọc Anh}
% \newcommand{\studentclass}{Toán Tin 01-K60}
% \newcommand{\studentcode}{20151366}


\usepackage[utf8]{inputenc}
\usepackage[utf8]{vietnam}
\usepackage{amsmath} % Required to use mathematical features
\usepackage{amsfonts}
\usepackage{amssymb}
\usepackage{geometry}
\geometry{
	a4paper,
	tmargin=35mm,
	bmargin=30mm,
	lmargin=35mm,
	rmargin=20mm,
}
\usepackage{mathptmx} % the replacement of Times New Roman
\linespread{1.25} % same as 1.5 line spacing in MS Word
\usepackage{graphicx}
\graphicspath{{images/}}
\usepackage{enumerate}
\usepackage{indentfirst}
\usepackage{fancybox}
\usepackage{fancyhdr}

\pagestyle{fancy}
\fancyhf{}
\lhead{\textit{\chaptername\space\thechapter}}
\rhead{\textit{GVHD: \teachername}}
% \lfoot{\textit{SVTH: \studentname}}
% \rfoot{\textit{\studentclass}}
\cfoot{\textit{\thepage}}
\fancyfoot[C]{\textit{\thepage}}
\renewcommand{\headrulewidth}{1pt}
\renewcommand{\footrulewidth}{1pt}

\fancypagestyle{plain}{
	\fancyhf{} % clear all header and footer fields
	\fancyfoot[C]{\textit{\thepage}} % except the center
	\renewcommand{\headrulewidth}{0pt}
	\renewcommand{\footrulewidth}{0pt}
}

\usepackage[backend=bibtex,style=ieee, sorting=nty]{biblatex}
\addbibresource{project_ref.bib}


\begin{document}
\begin{titlepage}
	\thisfancypage{\setlength{\fboxsep}{10pt}\doublebox}{}
	\begin{center}
		\begin{Large}\bfseries
			TRƯỜNG ĐẠI HỌC BÁCH KHOA HÀ NỘI\par
			VIỆN TOÁN ỨNG DỤNG VÀ TIN HỌC\par
		\end{Large}
		\vspace{1.5cm}
		\includegraphics[scale=0.3]{images/logo_bk.jpg}\par
		\vspace{1.5cm}
		\begin{LARGE}
			\MakeUppercase{Báo cáo môn học}\par
		\end{LARGE}
		\begin{LARGE}
			\MakeUppercase{Các mô hình ngẫu nhiên và ứng dụng}\par
		\end{LARGE}
		\vspace{1.5cm}
		\begin{Large}
			\MakeUppercase{Đề tài: \projectname}
		\end{Large}
		\vspace{1cm}
		\begin{large}
			% ĐỒ ÁN \project\par
			% \vspace{0.5cm}
			% Chuyên ngành: \MakeUppercase{\specialized}\par
			% \vspace{0.5cm}
			% Chuyên sâu: \orientation\par
			\begin{flushleft}
				\hspace{2cm}
				Nhóm sinh viên thực hiện: \textbf{PHÙNG TRỌNG HIẾU}\par
				\hspace{2cm}
				\space\space\space\space\space\space
				\space\space\space\space\space\space
				\space\space\space\space\space\space
				\space\space\space\space\space\space
				\space\space\space\space\space\space
				\space\space\space\space\space\space
				\space\space\space\space\space\space
				\space
				\textbf{CAO ĐĂNG SAO}\par
				\hspace{2cm}
				\space\space\space\space\space\space
				\space\space\space\space\space\space
				\space\space\space\space\space\space
				\space\space\space\space\space\space
				\space\space\space\space\space\space
				\space\space\space\space\space\space
				\space\space\space\space\space\space
				\space
				\textbf{NINH NGỌC LUYÊN}\par
				\hspace{2cm}
				\space\space\space\space\space\space
				\space\space\space\space\space\space
				\space\space\space\space\space\space
				\space\space\space\space\space\space
				\space\space\space\space\space\space
				\space\space\space\space\space\space
				\space\space\space\space\space\space
				\space
				\textbf{NGUYỄN MẠNH CƯỜNG}\par
				\vspace{0.5cm}
				\hspace{2cm}
				Giảng viên hướng dẫn: \MakeUppercase{\textbf{\teachername}}\par
				% \hspace{2.5cm}
				% Sinh viên thực hiện: \MakeUppercase{\studentname}\par
				% \hspace{2.5cm}
				% Lớp: \studentclass\par
				% \hspace{2.5cm}
				% MSSV: \studentcode\par
			\end{flushleft}
		\end{large}
		\vfill
		\begin{large}\bfseries
			HÀ NỘI - \the\year\par
		\end{large}
	\end{center}
\end{titlepage}

\frontmatter

\chapter*{\centering\LARGE\MakeUppercase{Nhận xét của giảng viên hướng dẫn}}
\begin{enumerate}
	\item Mục đích và nội dung của đồ án:\par
	\dotfill\par\dotfill\par\dotfill\par\dotfill\par\dotfill
	\item Kết quả đạt được:\par
	\dotfill\par\dotfill\par\dotfill\par\dotfill\par\dotfill
	\item Ý thức làm việc của sinh viên:\par
	\dotfill\par\dotfill\par\dotfill\par\dotfill\par\dotfill
\end{enumerate}
\hfill
\begin{minipage}[t]{0.48\textwidth}
	\begin{center}
		Hà Nội, ngày\hspace{5mm}tháng\hspace{5mm}năm\par
		Giảng viên hướng dẫn\par
		\textit{(Ký và ghi rõ họ tên)}
	\end{center}
\end{minipage}

\renewcommand{\listfigurename}{Danh mục hình vẽ}
\listoffigures
% \listoftables

\chapter{Lời cảm ơn}
% Để hoàn thành được đồ án: "Mô hình mạng nơ-ron học sâu và bài toán phát hiện vật thể trong ảnh",
% lời đầu tiên em xin chân thành cảm ơn thầy giáo hướng dẫn \teachername\space
% đã tận tình hướng dẫn em trong suốt thời gian thực hiện.

% Em cũng xin chân thành cảm ơn thầy giáo Lê Chí Ngọc đã cho em nhiều lời khuyên quý báu
% cùng các thầy cô trong viện Toán ứng dụng và Tin học Trường Đại học Bách Khoa Hà Nội
% đã truyền đạt những kinh nghiệm, kỹ năng, kiến thức nền tảng để giúp em hoàn thành đồ án môn học này.

% Xin cảm ơn các anh, các chị của phòng Machine Learning tại Công Ty TNHH Công Nghệ Cao SkymapGlobal Việt Nam
% đã hỗ trợ em về trang thiết bị và công nghệ.

% Cảm ơn tất cả các bạn trong tập thể lớp Toán Tin 1 và 2 - K60
% đã cho mình những ý kiến đóng góp và giúp đỡ trong quá trình làm đồ án.

% Tuy đã có những cố gắng nhất định, tìm hiểu và tiếp cận với đề tài
% nhưng do trình độ và thời gian hạn chế
% nên đồ án này không thể tránh khỏi các thiếu sót.
% Rất mong nhận được những nhận xét góp ý và sửa sai của thầy giáo hướng dẫn,
% các thầy cô, các đồng nghiệp và các bạn để quyển đồ án này trở nên hoàn thiện hơn.

Em xin chân thành cảm ơn!

\hfill
\begin{minipage}[t]{0.48\textwidth}
	\begin{center}
		Hà Nội, ngày \the\day\space tháng \the\month\space năm \the\year\par
		Trưởng nhóm\par
		Phùng Trọng Hiếu
	\end{center}
\end{minipage}

\tableofcontents

\mainmatter

\chapter{Giới thiệu}
\label{ch:intro}
	Xã hội ngày càng hiện đại, các kỹ thuật công nghệ ngày càng phát triển, đi cùng với nó là các nghiên cứu phát triển không ngừng về lĩnh vực trí tuệ nhân tạo và học máy, cho ra đời các hệ thống máy móc thông minh ứng dụng rộng rãi trong hầu hết các lĩnh vực đời sống như máy truy tìm dữ liệu, chẩn đoán y khoa, phát hiện thẻ tín dụng giả, phân tích thị trường chứng khoán, phân loại chuỗi DNA, nhận dạng tiếng nói và chữ viết, ... đặc biệt là trong lĩnh vực điều khiển.
	
	Chúng ta có rất nhiều loại thuật toán học như học có giám sát, học không có giám sát, học tăng cường, ... Mỗi loại thuật toán thích ứng với từng loại bài toán cụ thể. Trong báo cáo này, chúng ta sẽ nghiên cứu và tìm hiểu các vấn đề liên quan đến phương pháp học tăng cường (\textit{Reinforcement Learning}). Đây là một thuật toán học có khả năng giải quyết được những bài toán thực tế khá phức tạp trong đó có sự tương tác giữa hệ thống và môi trường. Với những tình huống môi trường không chỉ đứng yên, cố định mà thay đổi phức tạp thì các phương pháp học truyền thống không còn đáp ứng được mà phải sử dụng phương pháp học tăng cường. Những bài toán với môi trường thay đổi trong thực tế là không nhỏ và ứng dụng nhiều trong các lĩnh lực quan trọng.
	
	Môi trường thường được biểu diễn dưới dạng một quá trình quyết định Markov trạng thái hữu hạn (\textit{Markov Decision Process - MDP}), và các thuật toán học tăng cường cho ngữ cảnh này có liên quan nhiều đến các kỹ thuật quy hoạch động. Các xác suất chuyển trạng thái và các xác suất thu lợi trong MDP thường là ngẫu nhiên nhưng lại là tĩnh trong quá trình của bài toán. MDP được biết đến sớm nhất là vào những năm 1950 (cf. Bellman 1957~\cite{Belman1957}). Một cốt lõi của nghiên cứu về quá trình ra quyết định Markov là từ kết quả của cuốn sách của Ronald A.Howard (1960)~\cite{Howard1960}. Lý thuyết MDP được nghiên cứu bởi nhiều các nhà toán học, như  Bertsekas (2005)~\cite{Bertsekas2005} , White (1969)~\cite{White1969}, và Puterman (1994)~\cite{Puterman1994}. MDP cũng được nghiên cứu dưới dạng điều khiển tối ưu ngẫu nhiên, trong đó các phương pháp điều khiển tối ưu  có liên quan chặt chẽ với việc học tăng cường.
	
		Ví dụ đầu tiên mà  trong đó việc học tăng cường được thảo luận bằng cách sử dụng  MDP là Andreae’s (1969b)~\cite{Andreae1969b} mô tả  quan điểm về máy học. Witten and Corbin (1973)~\cite{Witten1973} đã thử nghiệm hệ thống học tăng cường, sau đó được phân tích bởi Witten (1977)~\cite{Witten1977} bằng cách sử dụng MDP. Mặc dù không đề cập rõ ràng về MDP, nhưng Werbos (1977)~\cite{Werbos1977} đã đề xuất các phương pháp giải xấp xỉ cho các bài toán điều khiển tối ưu ngẫu nhiên, có liên quan tới các phương pháp học tăng cường hiện đại. Chúng được sử dụng trong rất nhiều các lĩnh vực khác nhau, bao gồm robot, điều khiển tự động, kinh tế và chế tạo.\\
	
		
	Nội dung của báo cáo này được trình bày trong bốn chương. 
\begin{itemize}
	\item Chương 2: Quá trình Markov.
	
	Chương này trình bày một số định nghĩa về Xích Markov và Quá trình Markov.
	
	\item Chương 3: Quá trình quyết định Markov.
	
	Chương này trình này khái niệm quá trình quyết định Markov, bài toán Markov và các phần tử của bài toán Markov, phương trình tối ưu Bellman cho bài toán MDP.
	
	\item Chương 4: Thuật toán Q-Learning và Deep Q-Learning.
	
	Chương này giới thiệu về thuật toán Q-Learning và thuật toán Deep Q-Learning, sự hội tụ của thuật toán Q-Learning.
	
	\item Chương 5: Một số bài toán và ứng dụng cụ thể.
	
	Chương này trình bày hai ví dụ cụ thể về ứng dụng thuật toán Q-Learning và Deep Q-Learning trong giải quyết hai bài toán chiếc taxi thông minh và cân bằng con lắc ngược.
\end{itemize}


\chapter{Chương 2}
\label{ch:02}



\chapter{Quá trình quyết định Markov - MDP}
\label{ch:03}
Quá trình quyết định Markov (MDP) cung cấp một nền tảng toán học cho việc mô hình hóa việc ra quyết định trong các tinh huống mà kết quả là một phần ngẫu nhiên và một phần dưới sự điều khiển của một người ra quyết định. MDP rất hữu dụng cho việc học một loạt bài toán tối ưu hóa được giải quyết thông qua quy hoạch động và học tăng cường. MDP được sử dụng rất nhiều các lĩnh vực khác nhau, bao gồm robot, điều khiển tự động, kinh tế, chế tạo,... Trong phần này, chúng ta sẽ trình bày về quá trình quyết định Markov trong đó tập trung vào các khái niệm của quá trình Markov có số bước vô hạn và hữu hạn. Kiến thức trong phần này được tổng hợp từ các tài liệu \cite{Belman1957}, \cite{Sutton1999} và \cite{Puterman1994}.
\section{Khái niệm}
Một quá trình quyết định Markov (\textit{Markov Decision Process}) là một quá trình phần thưởng Markov (\textit{Markov Reward Process}) với các quyết định. Trong đó, tất cả các trạng thái đều thỏa mãn tính Markov. Cụ thể:
\begin{dn}\rm
(Quá trình quyết định Markov) Một quá trình quyết định Markov là một bộ năm thành phần $(S,A,P(.|.,.),R(.,.),\gamma)$, trong đó:
\begin{itemize}
 \item $S$ là một tập hữu hạn các trạng thái, kí hiệu $S_t$ là trạng thái tại thời điểm $t$; 
 \item $A$ là một tập hữu hạn các hành động, kí hiệu $A_t$ là hành động tại thời điểm $t (A(s)$ là tập hữu hạn các hành động có sẵn từ trạng thái $s$ với $s\in S)$;
 \item $Pr_{a}(s,s')=Pr(S_{t+1}=s'|S_t=s,A_t=a)$ là xác suất mà hành động $a$ tại trạng thái $s$ ở thời điểm $t$ chuyển sang trạng thái $s'$ tại thời điểm $t+1$. Xác suất này thỏa mãn tính Markov, nghĩa là:
 \begin{align*}
 Pr(S_{t+1}=j|S_0,A_0,...,S_t,A_t)=Pr(S_{t+1}=j|S_t,A_t);
 \end{align*}
 \item $R_a(s,s')$ là phần thưởng nhận được khi chọn hành động $s$ để chuyển trạng thái từ $s$ sang $s'$;
 \item $\gamma \in [0,1]$ là hệ số chiết khấu, đại diện cho sự khác biệt giữa các phần thưởng trong tương lai và phần thưởng hiện tại.
\end{itemize}
\begin{figure}[ht]
    \centering
    \begin{center}
    \includegraphics[scale=0.8]{ViduDnMDP.png}
    \end{center}    
    \caption{Quá trình chuyển trạng thái của sinh viên.}
    \label{fig:ViduDnMDP}
\end{figure}
\end{dn}
\begin{vd} \rm: Máy bán hàng tự động
\begin{itemize}
\item Trạng thái: cấu hình các khe.
\item Hành động: thời gian dừng lại.
\item Mục tiêu: kiếm được nhiều tiền.
\item Bài toán: tìm $\pi:S\to A$ sao cho $R$ lớn nhất.
\end{itemize}
\end{vd}
\begin{dn}\rm
(Chiến lược quyết định). Một\textit{ chiến lược }$\pi$ là một phân phối các hành động tại mỗi trạng thái:
\begin{align*}
\pi (a|s)=Pr[A_t=a|S_t=s].
\end{align*}
Một số kiểu chiến lược như sau:
\begin{itemize}
\item $\pi=(\pi^0,...,\pi^{T-1})$ được gọi là \textit{một chiến lược Markov} nếu tại mỗi thời điểm $n$, quyết định $\pi^n$ không phụ thuộc vào quá khứ $h_{n-1}$.\\
 Ví dụ: \begin{align*}
 \pi_{h_{n-1,i_n}}^{n}(a)=\pi_{h'_{n-1,i_n}}^{n}(a),~ a \in A(i_n), ~i_n \in S ~\forall ~h_{n-1},~h'_{n-1}.
 \end{align*}
 \item $\pi$ được gọi là một chiến lược dừng nếu mọi quy tắc quyết định đều bằng nhau :
 \begin{align*}
 \pi^n=\pi^0~~\text{với}~~n=0,1,...,T-1.
 \end{align*}
\end{itemize}
Trong quá trình quyết định Markov, mỗi chiến lược xác định đầy đủ hành động của một tác nhân. Ngoài ra, các chiến lược này chỉ phụ thuộc vào trạng thái của xích Markov tại thời điểm hiện tại (không phụ thuộc vào quá khứ).
\end{dn}
\section{Bài toán quyết định Markov}
\subsection{Phát biểu bài toán}
Bài toán quyết định Markov là bài toán học từ các tác động để đạt được mục đích. Người học và người ra quyết định được gọi là tác tử. Tất cả những gì mà chúng tương tác với, bao gồm mọi thứ bên ngoài tác tử được gọi là môi trường. Các tác động thực hiện một cách liên tục, tác tử lựa chọn hành động, môi trường đáp ứng lại các hành động đó và chuyển từ trạng thái hiện thời sang trạng thái mới. Môi trường cũng đem lại các mục tiêu, các giá trị hằng số mà tác tử cố gắng cực đại hóa qua thời gian. Một đặc tả hoàn thiện về môi trường được coi là một "nhiệm vụ", một thực thể của bài toán quyết định Markov.\\
Tóm lại, bài toán quyết định Markov liên quan đến lớp bài toán trong đó một tác tử rút ra kết luận trong khi phân tích một chuỗi các hàng động của nó cùng với tín hiệu vô hướng được đưa ra bởi môi trường.\\
Trong khái niệm chung này, có thể thấy hai đặc tính quan trọng:
\begin{itemize}
\item Tác tử tương tác với môi trường và cặp "tác tử + môi trường" tạo thành một hệ thống động.
\item Tín hiệu tăng cường, được nhận biết dựa vào mục tiêu, cho phép tác tử thay đổi hành vi của nó.
\end{itemize}
Lược đồ tương tác tác tử-môi trường như sau:\\
\newpage
\begin{figure}[ht]
    \centering
    \includegraphics[scale=0.7]{tactumoitruong.png}
    \caption{Mô hình tương tác giữa tác tử và môi trường.}
    \label{fig:tactumoitruong}
\end{figure}

Trong lược đồ trên, tác tử và môi trường tác động lẫn nhau tại mỗi bước trong chuỗi các bước thời gian rời rạc, $t=0,1,2,3,...$. Tại mỗi bước thời gian $t$, tác tử nhận một số biểu diễn về trạng thái của môi trường, $s_t \in S$, với $S$ là tập các trạng thái có thể, và trên đó lựa chọn một hàng động $a_t \in A(s_t)$, với $A(s_t)$ là tập các hành động trong trạng thái $s_t$. Mỗi bước thời gian tiếp theo, tác tử nhận một giá trị tăng cường $r_{t+1}\in R$ và tự nó tìm ra một trạng thái mới $S_{t+1}$.
\newline \medskip
Tại mỗi bước tác tử thực hiện ánh xạ từ các trạng thái đến các hành động có thể lựa chọn. Phép ánh xạ này được gọi là chiến lược của tác tử, kí hiệu là $\pi_t$ với $\pi_t(s,a)$ là xác suất thực hiện hành động $a_t=a$ khi $s_t=s$. 

\subsection{Các phần tử của bài toán quyết định Markov}
Dựa vào tác tử và môi trường, chúng ta có thể định nghĩa các phần tử con của một bài toán quyết định Markov: chiến lược (\textit{policy}), hàm phản hồi (\textit{reward function}), hàm giá trị (\textit{value function}), và không bắt buộc, một mô hình về môi trường.
\subsubsection*{Chiến lược}
\textit{Chiến lược} định nghĩa cách thức tác tử học từ hành động tại thời điểm đưa ra. Chiến lược là một ánh xạ từ tập các trạng thái của môi trường đến tập các hành động được thực hiện khi môi trường ở trong các trạng thái đó. Nó tương ứng với tập các luật nhân quả trong lĩnh vực tâm lý học. Trong một số trường hợp, chiến lược có thể là một hàm đơn giản hoặc một bảng tra cứu, trong những trường hợp khác, nó có thể liên quan đến các tính toán mở rộng, ví dụ như một tiến trình tìm kiếm. Chiến lược là nhân của một tác tử với nhận thức rằng một mình nó đủ quyết định hành động.
\newpage
\subsubsection{Hàm phản hồi}
Mục đích của tác tử là cực đại hóa các mục tiêu được tích lũy trong tương lai. Hàm phản hồi $R(t)$ được biểu diễn dưới dạng hàm số đối với các mục tiêu. Trong các bài toán quyết định Markov, hàm phản hồi sử dụng biểu thức dạng tổng. Các nhà nghiên cứu đã tìm ra ba biểu diễn thường được sử dụng của hàm phản hồi:
\begin{itemize}
\item \textit{Trong các bài toán số bước hữu hạn}

Với những bài toán này ta có một số hữu hạn các bước trong tương lai. Sẽ tồn tại một trạng thái kết thúc và một chuỗi các hành động giữa trạng thái đầu tiên và trạng thái kết thúc được gọi là một giai đoạn.
Ta có:
\begin{align*}
R(t)=r_{t}+r_{t+1}+...+r_{t+K-1}
\end{align*}
	Trong đó K là số các bước trước trạng thái kết thúc.
\item \textit{Trong các bài toán số bước vô hạn}\\
	Với những bài toán này ta có chuỗi các hành động là vô hạn. Một hệ số suy giảm $\gamma,0\leq \gamma \leq 1$ được đưa ra và hàm phản hồi được biểu diễn dưới dạng tổng của các giá trị mục tiêu giảm dần:
\begin{align*}
R(t)= \sum_{k=0}^{\infty} \gamma ^{k} r_{t+k}.
\end{align*}
Hệ số $\gamma$ cho phép xác định mức độ ảnh hưởng của những bước chuyển trạng thái tiếp theo đến giá trị phản hồi tại thời điểm đang xét. Gái trị của $\gamma$ cho phép điều chỉnh giai đoạn tác tử lấy các hàm tăng cường. Nếu $\gamma =0$, thì tác tử chỉ xem xét mục tiêu gần nhất, giá trị $\gamma$ càng gần với 1 thì tác tử sẽ quan tâm đến các mục tiêu xa hơn trong tương lai.
Như vậy, thực chất bài toán quyết định Markov trong trường hợp này chính là việc lựa chọn các hành động để làm cực đại biểu thức R:
\begin{align*}
R=r_{0}+\gamma r_{1}+\gamma^{2}r_{2}+... \textit{với} 0<\gamma <1.
\end{align*}
	Như trong hình vẽ minh họa sau:\\
	
\begin{figure}[ht]
    \centering
    \includegraphics[scale=0.8]{ttmoitruongvohan.png}
    \caption{Mô hình tương tác giữa tác tử và môi trường trong bài toán có số bước vô hạn.}
    \label{fig:tactumoitruong}
\end{figure}
\newpage
\item \textit{Trong các bài toán số bước vô hạn mà hàm phản hồi không hội tụ}

Trường hợp này xảy ra khi $\gamma = 1$. Giá trị trung bình của hàm phản hồi trên một bước thực hiện có thể hội tụ khi số bước tiến tới vô hạn. Trong trường hợp này hàm phản hồi được xác định bằng cách lấy trung bình của các giá trị tăng cường trong tương lai:
\begin{align*}
R(t)=\lim_{n\rightarrow \infty} \frac{1}{n}\sum_{k=0}^{\infty}r_{t+k}.
\end{align*}
\end{itemize}

\subsubsection{Hàm giá trị}
	Trong mọi trạng thái $s_{t}$, một tác tử lựa chọn một hành động dựa theo một chiến lược điều khiển, $ \pi: a_{t}=\pi(s_{t})$. Hàm giá trị tại một trạng thái của hệ thống được tính bằng kỳ vọng toán học của hàm phản hồi theo thời gian. Hàm giá trị là hàm của trạng thái và xác định mức độ thích hợp của chiến lược điều khiển $\pi$ đối với tác tử khi hệ thống đang ở trạng thái $s$. Hàm giá trị của trạng thái $s$ trong chiến lược $\pi$ được tính như sau:
\begin{align*}
V^{\pi}(s)=E_{\pi} \lbrace R_{t}|s_{t}=s\rbrace.
\end{align*}
	Bài toán tối ưu bao gồm việc xác định chiến lược điều khiển $\pi ^{*}$ sao cho hàm giá trị của trạng thái hệ thống đạt cực đại sau một số vô hạn hoặc hữu hạn các bước:
	$$ \pi^{*}= \lbrace \pi_{0}(s_{0}),\pi_{1}(s_{1}),...,\pi_{N-1}(s_{N-1} \rbrace.$$
	Đối với bài toán có số bước vô hạn ta có hàm giá trị trạng thái:
	$$ V^{\pi}(s)=E_{\pi} \lbrace R_{t}=\sum_{k=0}^{\infty}\gamma^{k}r_{t+k+1}|s_{t}=s \rbrace. $$
	Sử dụng các phép biến đổi:
	\begin{align*}
	V^{\pi}(s)&= E_{\pi} \lbrace R_{t}|s_{t}=s \rbrace \\
	&=E_{\pi} \lbrace \sum_{k=0}^{\infty} \gamma^{k} r_{t+k+1}|s_{t}=s \rbrace \\
	&=E_{\pi} \lbrace r_{t+1}+\gamma\sum_{k=0}^{\infty} \gamma^{k}r_{t+k+2}|s_{t}=s \rbrace\\
	&=\sum_{a} \pi(s,a) \sum_{s'} P_{ss'}^{a} [R_{ss'}^{a}+\gamma E_{\pi} \lbrace \sum_{k=0}^{\infty}\gamma^{k}r_{t+k+2}|s_{t+1}=s' \rbrace ]\\
	&=\sum_{a}\pi(s,a) \sum_{s'} P_{ss'}^{a}[R_{ss'}^{a}+\gamma V^{\pi}(s')].
	\end{align*}
	Như vậy, hàm $V^\pi (s)$ có thể được viết lại một cách đệ qui như sau:
$$V^\pi(s)=E_{\pi} {r_{t+1}+\gamma V^\pi (S_{t+1})|S_t=s},$$

hay
\begin{align}\label{1.11}
V^\pi(s)=R(s,a)+\gamma \sum_{s' \in S} P_{ss'}^{a}V^\pi (s').
\end{align}
Với $P_{ss'}^{a}$ là xác suất để chuyển từ trạng thái $s$ sang $s'$ khi áp dụng hành động $a$. Chúng ta có thể tính toán hàm $V^\pi(s)$ ngoại tuyến nếu biết trạng thái bắt đầu và xác suất mọi phép chuyển đổi theo mô hình. Vấn đề đặt ra là sau đó giải quyết hệ thống các phương trình tuyến tính trong công thức \ref{1.11}. Chúng ta biết rằng tồn tại một chiến lược tối ưu, kí hiệu $\pi^*$, được định nghĩa như sau :
\begin{align*}
&V^{\pi^*}(s) \geq V^\pi(s)\\
&\pi^* =argmax_\pi {V^\pi(s)},
\end{align*}
để đơn giản chúng ta viết $V^*=V^{\pi^*}$. Hàm giá trị tối ưu của một trạng thái tương ứng với chiến lược tối ưu là:
$$V^\pi(s)=max_\pi {V^\pi(s)},$$
đây là phương trình tối ưu Bellman (hoặc phương trình của quy hoạch động) ta sẽ nói chi tiết hơn ở phần tiếp theo.\\

Tóm lại $V^\pi$ \textit{là hàm giá trị trạng thái cho chiến lược $\pi$}. Giá trị của trạng thái kết thúc thường bằng $0$. Tương tự, định nghĩa $Q^\pi(s,a)$ là giá trị của việc thực hiện hành động $a$ trong trạng thái $s$ dưới chiến lược điều khiển $\pi$, được tính bằng kỳ vọng toán học của hàm phản hồi bắt đầu từ trạng thái $s$, thực hiện hành động $a$ trong chiến lược $\pi$:
\begin{align*}
.Q^\pi(s,a)=E_\pi \lbrace R_t|s_t=s,a_t=a\rbrace =E_\pi \Big\lbrace \sum_{k=0}^{\infty}\gamma^k r_{t+k+1}|s_t=s,a_t=a \Big \rbrace,
\end{align*}
$Q^\pi$ được gọi là hàm giá trị hành động cho chiến lược $\pi$. Và các hàm giá trị $V^\pi$, $Q^\pi$ có thể được ước lượng từ kinh nghiệm.
\section{Phương trình tối ưu Bellman cho bài toán MDP}

Từ trạng thái $s$, có thể đưa ra nhiều hành động khác nhau và mỗi chiến lược xác định một phân phối xác suất của hành động đó, do đó, sử dụng phương trình tối ưu Bellman để đưa ra quyết định cho bài toán này.\\
Phương trình Bellman cho hàm giá trị trạng thái:
\begin{align*}
V^{\pi}(s)&=R_{\pi}(s)+\gamma \sum_{s'\in S}P_{\pi}(s,s')V^{\pi}(s')\\
&=\sum_{a\in A}\pi(a|s)\left ( R_{a}(s)+\gamma \sum_{s'\in S}P_{\pi}(s,s')V^{\pi}(s') \right ).
\end{align*}
Phương trình Bellman cho hàm giá trị hành động:
\begin{align*}
Q^{\pi}(s,a)=R_{a}(s)+\gamma \sum_{s'\in S} P_{a}(s,s')\sum_{a' \in A}\pi(a'|s')Q^{\pi}(s',a').
\end{align*}
\begin{dn} \rm
Hàm giá trị trạng thái tối ưu $V^{*}(s)$ là hàm trả về trạng thái tối ưu trên tất cả các chiến lược:
\begin{align*}
V^{*}(s)~=~\displaystyle\max_{\pi} V^{\pi}(s).
\end{align*}
\end{dn}
\begin{dn} \rm
Hàm giá trị hành động tối ưu $Q^{*}(s,a)$ là hàm trả về các hành động tối ưu trên tất cả các chiến lược:
\begin{align*}
Q^{*}(s,a)~=~\displaystyle\max_{\pi} Q^{\pi}(s,a).
\end{align*}
\end{dn}

\begin{dn}\rm 
(Chiến lược tối ưu) $\pi$ được gọi là chiến lược tối ưu nếu $V^{\pi}(s) \geq V^{\pi '}(s), \forall s.$\\
Xác định một chiến lược tối ưu: Một chiến lược tối ưu có thể được xác định bằng cách tìm hàm cực đại $Q^{*}(s,a)$:
\begin{align}
\pi^{*}(a|s)=\begin{cases}
1 & \text{ nếu } a=argmax_{a\in A}~Q^{*}(s,a) \\ 
0 & \text{ ngược lại } 
\end{cases}
\end{align}
Phương trình tối ưu Bellman cho $V^{*}(s)$:
\begin{align}
V^{*}(s)= \displaystyle\max_{\pi}~ R_{a}(s)+\gamma \sum_{s'\in S}P_{a}(s,s')V^{*}(s').
\end{align}
Phương trình tối ưu Bellman cho $Q^{*}(a,s)$:
\begin{align}
Q^{*}(s,a)=R_{a}(s)+\gamma \sum_{s'\in S}P_{a}(s,s') \displaystyle\max_{a'}~Q^{*}(s',a').
\end{align}
Phương trình tối ưu Bellman là một phương trình phi tuyến có thể giải bằng một số phương pháp: lặp giá trị và lặp chính sách.
\end{dn}

\chapter{Chương 4}
\label{ch:04}



\chapter{Một số bài toán ứng dụng}
\label{ch:05}

Trong chương này, ta sẽ cùng nhau giải quyết
hai bài toán học tăng cường cơ bản để minh họa cho việc
sử dụng thuật toán Q-Learning đã được trình bày ở chương trước.
Bài toán đầu tiên sẽ là bài Chiếc taxi thông minh (Smart Taxi);
trong bài toán này, ta sẽ huấn luyện một chiếc taxi
sao cho nó có thể đón và trả khách tại đúng vị trí
và thực hiện việc này một cách "thông minh" nhất có thể.
Bài toán thứ hai sẽ giải quyết việc điều khiển
chiếc xe đẩy để giữ cho con lắc được gắn trên xe
luôn ở trạng thái cân bằng; bài toán kinh điển này còn
được biết đến với cái tên Bài toán cân bằng con lắc ngược (CartPole).

\section{Bài toán chiếc taxi thông minh}
\subsection{Bài toán}
Ta có một chiếc taxi được trang bị các cảm biến, trí tuệ nhân tạo,~\dots\space
để có thể tự vận hành trong mọi điều kiện giao thông và thời tiết.
Nhiệm vụ của chiếc xe này là đón và trả khách tại những vị trí nhất định.
Ngoài ra, việc vận chuyển hành khách cần phải thỏa mãn những tiêu chí sau:
\begin{itemize}
    \item Phải trả khách tại đúng vị trí được chỉ định
    \item Tiết kiệm thời gian cho hành khách một cách tối đa
    \item Đảm bảo hành khách được an toàn và phải tuân thủ tất cả các luật giao thông được đưa ra
\end{itemize}

\subsection{Mô hình hóa bài toán}
Trước khi có thể sử dụng các kỹ thuật học tăng cường
để huấn luyện cho chiếc taxi (agent của chúng ta)
thực hiện công việc đưa đón khách một cách tự động,
ta cần phải quan tâm đến một vài khía cạnh
về việc mô hình hóa bài toán.
Ta cần phải biết phần thưởng (rewards),
không gian trạng thái (state space) của chiếc taxi,
và các hành động (actions) mà chiếc taxi
có thể thực hiện tại mỗi trạng thái.

\subsubsection{Phần thưởng}
Vì chiếc taxi sẽ được huấn luyện bằng cách
thử và sai khi tương tác với môi trường,
ta cần phải định nghĩa phần thưởng và/hoặc hình phạt cho nó:
\begin{itemize}
    \item Chiếc taxi (agent) sẽ nhận được một phần thưởng lớn (+20 điểm)
    khi trả khách thành công (trả đúng vị trí được đưa ra)
    \item Chiếc taxi sẽ bị phạt nặng nếu nó trả khách sai vị trí (-10 điểm)
    \item Chiếc taxi sẽ bị phạt "nhẹ" (slight negative reward)
    trong suốt chuyến hành trình đi đến vị trí trả khách (-1 điểm/bước).
    Hình phạt ở đây không được lớn vì ta không muốn
    việc chiếc taxi cố gắng "lao" đến đích một cách nhanh nhất có thể
    mà vi phạm luật giao thông hay gây nguy hiểm cho hành khách.
\end{itemize}

\subsubsection{Không gian trạng thái}
Không gian trạng thái là tập chứa tất cả những tình huống
mà chiếc taxi của chúng ta có thể gặp phải.
Đây là nơi chứa những thông tin vô cùng cần thiết
cho chiếc taxi để giúp nó có thể đưa ra những hành động "đúng"
tương ứng với trạng thái mà nó đang ở.

Giả sử, ta có một bãi tập cho chiếc taxi của chúng ta,
như được minh họa trong Hình~\ref{fig:training_area};
ở đây, ta sẽ dạy chiếc taxi vận chuyển hành khách
đến các vị trí (R, G, Y, B) trên bãi tập.
\begin{figure}[H]
    \centering
    \includegraphics[scale=0.6]{rl_smart_taxi_env.png}
    \caption{Bãi tập.}
    \label{fig:training_area}
\end{figure}

Để đơn giản hóa bài toán, ta có một số giả định như sau:
\begin{itemize}
    \item Chiếc taxi là phương tiện duy nhất có trên bãi tập
    \item Khu vực huấn luyện có thể được chia thành một lưới $5 \times 5$,
    cho ta tổng cộng 25 vị trí mà chiếc taxi có thể đỗ.
    Ví dụ, như ta có thể thấy trên Hình~\ref{fig:training_area},
    chiếc taxi đang nằm tại vị trí có tọa độ (3, 1);
    ngoài ra 4 vị trí R, G, Y, B có tọa độ lần lượt là
    (0,0), (0,4), (4,0), (4,3);
    vị hành khách đáng kính đang đứng tại vị trí Y
    và có mong muốn di chuyển đến vị trí R trên bãi tập.
\end{itemize}

Vậy, ta có tổng cộng $5 \times 5=25$ vị trí mà chiếc taxi có thể xuất hiện,
4 đích đến, và 5 vị trí của hành khách
(4 vị trí tại R, G, Y, B và 1 vị trí là ở trên chiếc taxi).
Tổng số trạng thái có thể có của môi trường sẽ là
$5 \times 5 \times 5 \times 4=500$ trạng thái.

\subsubsection{Không gian hành động}
Tại mỗi thời điểm, chiếc taxi (agent của bài toán)
sẽ nằm ở 1 trong tổng 500 trạng thái,
và nó sẽ thực hiện một hành động tương ứng với trạng thái hiện có.
Hành động ở đây có thể là đón/trả khách, và di chuyển quanh bãi tập.

Không gian hành động của ta sẽ gồm:
\begin{itemize}
    \item Đi lên
    \item Đi xuống
    \item Đi sang trái
    \item Đi sang phải
    \item Đón khách
    \item Trả khách
\end{itemize}

Để ý rằng, tại một số trạng thái ta không thể thực hiện
một vài hành động nhất định.
Ví dụ như khi chiếc taxi ở vị trí mép tường bên trái,
nó không thể thực hiện hành động đi sang trái;
ta có thể giải quyết vấn đề này bằng việc phạt chiếc taxi
khi rơi vào tình huống đó (tình huống bị "đâm" và tường)
và giữ nguyên vị trí hiện tại của nó.

\subsection{Giải quyết bài toán}
Giải pháp được sử dụng để giải quyết bài toán này
sẽ là giải thuật Q-Learning.
Môi trường của bài toán sẽ được mô phỏng nhờ vào sự trợ giúp
của thư viện Gym được cung cấp bởi OpenAI.

Chiếc taxi sẽ được huấn luyện trong 100000 episode
thông qua việc tương tác với môi trường.
Một vài kết quả trong quá trình huấn luyện được minh họa như trong
Hình~\ref{fig:training_episode_step},
\ref{fig:training_episode_penalty} và \ref{fig:training_episode_reward}.

\begin{figure}[H]
    \centering
    \includegraphics[scale=0.7]{training_episode_step.png}
    \caption{Số bước chuyển chiếc taxi thực hiện tại mỗi episode.}
    \label{fig:training_episode_step}
\end{figure}
\begin{figure}[H]
    \centering
    \includegraphics[scale=0.7]{training_episode_penalty.png}
    \caption{Số lần chiếc taxi đón/trả khách sai vị trí tại mỗi episode.}
    \label{fig:training_episode_penalty}
\end{figure}
\begin{figure}[H]
    \centering
    \includegraphics[scale=0.7]{training_episode_reward.png}
    \caption{Số phần thưởng chiếc taxi nhận được tại mỗi episode.}
    \label{fig:training_episode_reward}
\end{figure}

Kết quả chạy của bài toán với 1000 episode
sau quá trình huấn luyện được minh họa như trong Hình~\ref{fig:prediction_results}

\begin{figure}[H]
    \centering
    \includegraphics[scale=0.7]{prediction_results.png}
    \caption{Kết quả chạy với 1000 episode.}
    \label{fig:prediction_results}
\end{figure}

Có thể thấy chiếc taxi đã được huấn luyện khá tốt,
không mắc bất cứ sai lầm nào trong việc đón/trả khách
và thực hiện việc chọn đường đi khá "thông minh".


\section{Bài toán cân bằng con lắc ngược}

\begin{figure}[H]
    \centering
    \includegraphics[scale=0.7]{cartpole.png}
    \caption{Hình minh họa bài toán cân bằng con lắc ngược.}
    \label{fig:cartpole}
\end{figure}

\subsection{Bài toán}
Ta có một con lắc ngược được gắn phía trên
một chiếc xe đẩy bằng một khớp nối có thể quay.
Chiếc xe được đặt trên một đường ray không có ma sát
và chỉ có thể di chuyển sang trái hoặc sang phải.
Ở trạng thái khởi đầu, con lắc được đặt thẳng đứng,
hay nói cách khác con lắc và đường thẳng vuông góc
với đường ray tạo thành một góc 0\degree.
Mục tiêu của bài toán là không để cho con lắc bị đổ
bằng cách tăng hoặc giảm vận tốc của xe đẩy.
Trò chơi sẽ kết thúc khi:
\begin{itemize}
    \item con lắc bị "đổ"
    (góc của con lắc nhỏ hơn -12\degree hoặc lớn hơn 12\degree),
    \item hoặc vị trí của xe đẩy năm ngoài đoạn $[-2.4, 2.4]$
    (tâm của chiếc xe đẩy đi ra ngoài khung nhìn),
    \item hoặc số bước vượt quá 200.
\end{itemize}

\subsection{Mô hình hóa bài toán}
\subsubsection{Phần thưởng}
Hệ chuyển động sẽ nhận được phần thưởng +1 sau mỗi bước
(bao gồm cả bước kết thúc).

\subsubsection{Không gian trạng thái}
Mỗi trạng thái của hệ sẽ được cấu tạo bởi 4 thành phần:
\begin{itemize}
    \item Vị trí xe đẩy
    \item Vận tốc xe đẩy
    \item Góc của con lắc ngược
    \item Vận tốc ở đỉnh con lắc
\end{itemize}

Miền giá trị của các thành phần này được minh họa như trong Bảng~\ref{tab:states}.

\begin{table}[H]
    \centering
    \begin{tabular}{|l|l|l|l|}
    \hline
    \multicolumn{1}{|c|}{Chỉ số} & \multicolumn{1}{c|}{Trạng thái} & \multicolumn{1}{c|}{Min} & \multicolumn{1}{c|}{Max} \\ \hline
    0                            & Vị trí xe đẩy                   & -2.4                     & 2.4                      \\ \hline
    1                            & Vận tốc xe đẩy                  & -Inf                     & Inf                      \\ \hline
    2                            & Góc của con lắc ngược           & $\sim$-41.8\degree       & $\sim$41.8\degree        \\ \hline
    3                            & Vận tốc ở đỉnh con lắc          & -Inf                     & Inf                      \\ \hline
    \end{tabular}
    \caption{Các trạng thái của hệ.}
    \label{tab:states}
\end{table}

\subsubsection{Không gian hành động}
Có tổng cộng 2 hành động để di chuyển con lắc,
đó là đẩy xe sang trái và sang phải.

\subsection{Giải quyết bài toán}
Với số trạng thái quá lớn, việc sử dụng giải thuật Q-Learning
như trong bài toán "Chiếc taxi thông minh" trước đó trở nên
bất khả thi về không gian lưu trữ.
Việc sử dụng một mạng nơ-ron nhân tạo (Artificial Neural Network)
thay thế cho bảng Q-Table có vẻ phù hợp hơn rất nhiều.

Hệ xe đẩy con lắc sẽ được huấn luyện trong 50000 episode
thông qua việc tương tác với môi trường.
Một vài kết quả trong quá trình huấn luyện được minh họa như trong
Hình~\ref{fig:cartpole_episode_reward} và \ref{fig:cartpole_running_avg_reward}.

\begin{figure}[H]
    \centering
    \includegraphics[scale=1]{cartpole_episode_reward.png}
    \caption{Số phần thưởng hệ nhận được tại mỗi episode.}
    \label{fig:cartpole_episode_reward}
\end{figure}

\begin{figure}[H]
    \centering
    \includegraphics[scale=1]{cartpole_running_avg_reward.png}
    \caption{Số phần thưởng trung bình hệ nhận được qua các episode.}
    \label{fig:cartpole_running_avg_reward}
\end{figure}

Dựa vào các biểu đồ, ta có thể thấy hệ dễ dàng đạt được
số điểm tối đa, 200 điểm, sau khoảng 10000 episode.

\chapter{Kết luận}
\label{ch:conc}



\backmatter

% \appendix
% \include{chapters/appendix}

\printbibliography[heading=bibintoc]

\end{document}